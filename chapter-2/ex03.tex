\begin{enunciado}{3}
    Compute the maximum number of dichotomies, $m_{\hipotset}(N)$, for these
    learning models, and consequently compute $\dvc$, the VC dimension.
    
    \letra{a} Positive or negative ray: $\hipotset$ contains the functions
    which are $+1$ on $[a, \infty)$ (for some $a$) together with those
    that are $+1$ on $(-\infty, a]$ (for some $a$).
    
    \letra{b} Positive or negative interval: $\hipotset$ contains the functions which are $+1$ on an interval $[a, b]$ and $-1$ elsewhere or $-1$ on an interval $[a, b]$ and $+1$ elsewhere.
    
    \letra{c} Two concentric spheres in $\conjR^d$: $\hipotset$ contains the
    functions which are $+1$ for $a \le \sqrt{x_1^2 + \dots + x_d^2} \le b$.
\end{enunciado}


\begin{enumerate}
    \item[(a)] raios positivos ou negativos: \\
    Neste exercício para calcular $m_H(N)$, notamos que dados $N$
    temos duas dicotomias por cada um dos $N-1$ intervalos internos, dependendo 
    da região na qual está $a$.\\ 
    Já no caso das regiões externas só tem 
    uma dicotomia para o caso de $a$ ficar na região $R_1$ em que todos são
    positivos por  $[-\infty, a)$. Algo semelhante acontece para quando $a$ está na região $R_{N+1}$
    só tem uma dicotomia em que todos são positivos por $[a,\infty)$. 
    Por tanto  $m_H(N) = 2(N-1)+2 = 2N$. Por outro lado a $d_{vc}(H)=2$, $m_H(3)= 2\times 3 = 6 \neq 2^3$\\

    \item[(b)] intervalos positivos ou negativos:\\
    De maneira semelhante à anterior com $N$ pontos temos $N+1$ regiões. A dicotomia
    que temos é decidir em quais regiões estão $a$ e $b$ o qual resulta em $\binom {N-1} {2}$ dicotomias.
    Mas por cada uma delas precisamos ainda de duas dicotomias (o intervalo é positivo ou negativo). 
    Já para o caso em que $a$ e $b$ estao na mesma região caimos no caso do exercício 2.3(a) o qual tem $2N$ 
    dicotomias.
    Por tanto $m_H(N) = 2\times \binom {N-1} +2(N) = 2 (\frac{(N-1)(N-2)}{2}) + 2N =
    N^2 - N + 2$.
    Por outro lado a $d_{vc}(4)= 16-4+2=14 \neq 2^4$
    A $d_{vc}(H)= 3$, porque nenhuma das funções em $H$ consegue etiquetar
    o intervalo +,-,+,- ou -,+,-,+.\\

    \item[(c)] esferas concêntricas em $\mathbb{R}^d$\\

    Neste caso vamos visualizar o exercício em função  das distancias euclideana desde o centro da esfera
    até $a$ e $b$ e até cada um dos pontos $(x_1,x_2, ..., x_d)$, e estabelecemos uma equivalência baseada na correlacao 
    com os número naturais. Desta maneira o exercício fica exatamente igual ao exemplo feito na aula 
    em que: $ m_H(N) =\frac{N^2}{2} + \frac{N}{2} + 1$
    A $d_{vc}(H)= 2$
    Podemos colocar qualquer conjunto de 3 pontos em $r_1 \leq a$,
    $a \leq r_2 \leq b$ e $b \leq r_3$ desde o origem. 
    E não existe uma função em $H$ que pode colocar $r_1$ e $r_2$ com etiqueta $+1$.
\end{enumerate}
