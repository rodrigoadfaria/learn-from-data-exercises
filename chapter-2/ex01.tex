\begin{enunciado}{1}
    In Equation (2.1), set $\delta = 0.03$ and let
    
    $$ \epsilon(M, N, \delta) = \sqrt{\frac{1}{2N} \ln{\frac{2M}{\delta}}}$$
    
    \letra{a} For M = 1, how many examples do we need to make $\epsilon \le 0.05$?
    
    \letra{b} For M = 100, how many examples do we need to make $\epsilon \le 0.05$?
    
    \letra{c} For M = 10,000, how many examples do we need to make $\epsilon \le 0.05$?
    
    \begin{grayt}
		$ E_{out}(g) \leq E_{in}(g) + \sqrt{\frac{1}{2N} \ln{\frac{2M}{\delta}}} $
    \end{grayt}
    
\end{enunciado}

\letra{a} Resposta

\begin{align*}
	\sqrt{\frac{1}{2N} \ln{\frac{2M}{\delta}}} \le 0.05 \\
	\sqrt{\frac{1}{2N} \ln{\frac{2(1)}{0.03}}} \le 0.05 \\
	\sqrt{\frac{1}{2N} {4.199705}} \le 0.05 \\
	% elevando ao quadrado para extrair a raiz
	\left( \sqrt{\frac{1}{2N} {4.199705}} \right) ^ {2} \le (0.05)^{2} \\
	\frac{4.199705}{2N} \le 0.0025 \\
	\frac{4.199705}{0.0025} \ge 2N \\
	% invertendo os lados
	N \ge \frac{1679.882}{2} \\
	N \ge {839.941} \\
\end{align*}


\letra{b} Resposta

\begin{align*}
	\sqrt{\frac{1}{2N} \ln{\frac{2M}{\delta}}} \le 0.05 \\
	\sqrt{\frac{1}{2N} \ln{\frac{2(100)}{0.03}}} \le 0.05 \\
	\sqrt{\frac{1}{2N} {8.804875}} \le 0.05 \\
	% elevando ao quadrado para extrair a raiz
	\left( \sqrt{\frac{1}{2N} {8.804875}} \right) ^ {2} \le (0.05)^{2} \\
	\frac{8.804875}{2N} \le 0.0025 \\
	\frac{8.804875}{0.0025} \ge 2N \\
	% invertendo os lados
	N \ge \frac{3521.950105}{2} \\
	N \ge {1760.975} \\
\end{align*}


\letra{c} Resposta

\begin{align*}
	\sqrt{\frac{1}{2N} \ln{\frac{2M}{\delta}}} \le 0.05 \\
	\sqrt{\frac{1}{2N} \ln{\frac{2(10000)}{0.03}}} \le 0.05 \\
	\sqrt{\frac{1}{2N} {12.716898}} \le 0.05 \\
	% elevando ao quadrado para extrair a raiz
	\left( \sqrt{\frac{1}{2N} {12.716898}} \right) ^ {2} \le (0.05)^{2} \\
	\frac{12.716898}{2N} \le 0.0025 \\
	\frac{12.716898}{0.0025} \ge 2N \\
	% invertendo os lados
	N \ge \frac{5086.759307}{2} \\
	N \ge {2543.379} \\
\end{align*}

