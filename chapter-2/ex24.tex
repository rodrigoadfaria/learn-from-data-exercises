\begin{enunciado}{24}
    Consider a simplified learning scenario. Assume that the input dimension is one.     Assume that the input variable $x$ is uniformly distributed in the interval         $[-1,1]$. The data set consists of $2$ points $\{x_{1}, x_{2}\}$ and assume that     the target function is $f(x) = x^{2}$. Thus, the full data set is $\dset = \{       (x_{1}, x_{1}^{2}), (x_{2}, x_{2}^{2})\}.$  The learning algorithm returns the      line fitting these two points as $g$ ($\hipotset$ consists of functions of the      form $h(x) = ax + b$). We are interested in the test performance ($E_{out}$) of     our learning system with respect to the squared error measure, the bias and the     var. \\
    
    \letra{a} Give the analytic expression for the average function $\bar{g}(x)$. \\
    
    \letra{b} Describe a $n$ experiment that you could run to determine                 (numerically) $\bar{g}(x)$, $E_{out}$ , bias, and var. \\
    
    \letra{c} Run your experiment and report the results. Compare $E_{out}$ with        bias+var. Provide a plot of your $\bar{g}(x)$ and $f(x)$ (on the same plot). \\
    
    \letra{d} Compute analytically what $E_{out}$,  bias and var should be.
\end{enunciado}

Resposta resposta resposta