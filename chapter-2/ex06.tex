\begin{enunciado}{6}
    Prove that for $N \ge d$,
    
    $$\sum\limits_{i=0}^{d} \binomial{N}{i} \le \left( \frac{eN}{d} \right)^d$$
    
    We suggest you first show the following intermediate steps.
    
    \letra{a} $\sum\limits_{i=0}^{d}\binomial{N}{i} \le 
    \sum\limits_{i=0}^{d}\binomial{N}{i} \left( \frac{N}{d} \right)^{d-i} \le
    \left( \frac{N}{d} \right)^{d} \sum\limits_{i=0}^{N}\binomial{N}{i} \left( \frac{d}{N} \right)^{i}$
    
    \letra{b} $\sum\limits_{i=0}^{N}\binomial{N}{i} \left( \frac{d}{N} \right)^{i} \le e^d$. \textit{[Hints: Binomial theorem; $(1 + \frac{1}{x})^x \le e$ for $x > 0$.]}
    
    Hence, argue that $m_{\hipotset}(N) \le (\frac{eN}{\dvc})^{\dvc}$.
\end{enunciado}

\textbf{(a)}

Como $N \ge d$, então $\frac{N}{d} \ge 1$. E como $i \le d$, $d - i \ge 0$ e portanto $\left(\frac{N}{i} \right)^{d-i} \ge 1$

Multiplicando cada termo do somatório $\sum\limits_{i=0}^{d} \binomial{N}{i}$ por um valor maior ou igual a $1$, podemos escrever a relação

$\sum\limits_{i=0}^{d} \binomial{N}{i} \le \sum\limits_{i=0}^{d} \binomial{N}{i}\left(\frac{N}{d}\right)^{d-i}$

Para $0 \le i \le d$, temos que $\left(\frac{N}{d}\right)^{i}\left(\frac{d}{N}\right)^{i} = 1$. Assim,

$\sum\limits_{i=0}^{d} \binomial{N}{i} \le \sum\limits_{i=0}^{d} \binomial{N}{i}\left(\frac{N}{d}\right)^{d-i}=
\sum\limits_{i=0}^{d} \binomial{N}{i}\left(\frac{N}{d}\right)^{d-i}
\left(\frac{N}{d}\right)^{i}\left(\frac{d}{N}\right)^{i} = 
\sum\limits_{i=0}^{d} \binomial{N}{i}\left(\frac{N}{d}\right)^{d}
\left(\frac{d}{N}\right)^{i}
$

Como $\left(\frac{N}{d}\right)^d$ não depende de $i$, podemos tirá-lo do somatório e obter a relação desejada

$\sum\limits_{i=0}^{d} \binomial{N}{i} \le \sum\limits_{i=0}^{d} \binomial{N}{i}\left(\frac{N}{d}\right)^{d-i} \le
\left(\frac{N}{d}\right)^{d} \sum\limits_{i=0}^{d} \binomial{N}{i}
\left(\frac{d}{N}\right)^{i}
$\cqd

\vspace{1cm}

\textbf{(b)}

O Teorema Binomial nos diz o seguinte:

$(1 + x)^n = \sum\limits_{i=0}^n \binomial{n}{i} x^i$


Observe que $\sum\limits_{i=0}^{N}\binomial{N}{i} \left( \frac{d}{N} \right)^{i}$ tem a forma adequada para usarmos esse teorema. Usando o teorema, com $x = \left( \frac{d}{N} \right)$, obtemos a igualdade

$\sum\limits_{i=0}^{N}\binomial{N}{i} \left( \frac{d}{N} \right)^{i} = (1 + \frac{d}{N})^N$

Resta mostrar que $\left( 1 + \frac{d}{N} \right)^N \le e^d$.

Seja $t \in [1, +\infty[$, então

$\frac{1}{t} \le 1$

Tomando a integral dos dois lados, de $1$ a $1 + \frac{d}{N}$, temos:

$$ \int_{1}^{1 + \frac{d}{N}} \frac{1}{t} \le
\int_{1}^{1 + \frac{d}{N}} 1$$

O que nos dá o seguinte

$$ \ln(1 + \frac{d}{N}) \le \frac{d}{N} $$

Exponenciando,

$$1 + \frac{d}{N} \le e^{\frac{d}{N}}$$

Elevando a $N$ dos dois lados, obtemos a relação desejada:

$$ \left( 1 + \frac{d}{N} \right)^N \le e^d$$

Assim, concluímos que
$\sum\limits_{i=0}^{N}\binomial{N}{i} 
\left( 
\frac{d}{N} \right)^{i} 
\le e^d$\cqd
