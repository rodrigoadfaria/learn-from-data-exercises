\begin{enunciado}{8}
    Which of the following are possible growth functions $m_{\hipotset}(N)$ for some hypothesis set:
    
    $$1+N; \quad 1+N+\frac{N(N-1)}{2}; \quad 
    2^N; \quad 2^{\floor{\sqrt{N}}}; \quad 
    2^{\floor{\frac{N}{2}}}; \quad
    1+N+\frac{N(N-1)(N-2)}{6}$$
\end{enunciado}

Temos o seguinte teorema sobre as funções de crescimento. Se $m(k)<2^k$ para algum (qualquer) $k$, então, para todo $N$, $m(N)\le N^{k-1}+1$.

Vamos analisar cada uma das funções propostas com base neste teorema.

A função $1+N$ pode ser uma função de crescimento, pois para $k = 2$ temos que $1 + k < 2^k$, ou seja, $1 + 2 < 2^2$  logo a proposição da primeira parte do teorema é válida logo, temos, pelo teorema que, para todo $N$, 

\begin{align*}
    m(N) &\leq N^{k-1}+1 \\
    & \leq N + 1
\end{align*}

Esta é a própria função proposta, logo é valida como função de crescimento.

Para a função $1+N+\frac{N(N-1)}{2}$, tomemos $k = 3$ em que $1+k+\frac{k(k-1)}{2} < 2^k$, ou seja, $1+3+\frac{3(3-1)}{2} < 2^3$, logo a proposição da primeira parte do teorema é válida logo, temos, pelo teorema que, para todo $N$, 

\begin{align*}
    m(N) &\leq N^{k-1}+1 \\
    & \leq N^2 + 1
\end{align*}

No exercício 1.5 provamos que $\sum\limits_{i=0}^{D} \binomial{N}{i} \le N^D + 1$. Temos que $\sum\limits_{i=0}^{k-1} \binomial{N}{i}$ para $k = 3$ é igual a $1+N+\frac{N(N-1)}{2}$, logo $1+N+\frac{N(N-1)}{2} \leq N^2 + 1$ e portanto é uma função válida como função de crescimento.

Para a função $2^N$ não é possível encontrar um valor de $k$ tal que $2^k < 2^k$ logo, $k = \infty$ o teorema acima não é valido e $2^N$ é a função de crescimento para $k = \infty$ e portanto, $2^N$ é uma função de crescimento válida.

Para a função $2^{\floor{\sqrt{N}}}$ temos que para $k = 4$, $2^{\floor{\sqrt{k}}} < 2^k$, ou seja, $2^{\floor{\sqrt{4}}} < 2^4$ e portanto a proposição da primeira parte do teorema é válida, logo, para todo $N$

\begin{align*}
    m(N) &\leq N^{k-1}+1 \\
    & \leq N^3 + 1
\end{align*}

A função $2^{\floor{\sqrt{N}}}$ não é menor ou igual a $N^3 + 1$ e portanto, não pode ser uma função de crecimento.

Para a função $2^{\floor{\frac{N}{2}}}$ temos que para $k = 4$, $2^{\floor{\frac{k}{2}}} < 2^k$, ou seja, $2^{\frac{4}{2}} < 2^4$ e portanto a proposição da primeira parte do teorema é válida, logo, para todo $N$

\begin{align*}
    m(N) &\leq N^{k-1}+1 \\
    & \leq N^3 + 1
\end{align*}

A função $2^{\frac{N}{2}}$ não é menor ou igual a $N^3 + 1$ e portanto, não pode ser uma função de crecimento.

Para a função $1+N+\frac{N(N-1)(N-2)}{6}$ temos que e para $k = 2$ temos que $1+k+\frac{k(k-1)(k-2)}{6} < 2^k$, ou seja $1+2+\frac{2(2-1)(2-2)}{6} < 2^2$,  logo a proposição do teorema é válida. Dessa forma, para todo $N$ 

\begin{align*}
    m(N) &\leq N^{k-1}+1 \\
    & \leq N + 1
\end{align*}

A função $1+N+\frac{N(N-1)(N-2)}{6}$ não é menor ou igual a $N + 1$ e portanto, não pode ser uma função de crecimento.
